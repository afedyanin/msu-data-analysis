\documentclass[11pt]{article}

    
\usepackage[breakable]{tcolorbox}
    \usepackage{parskip} % Stop auto-indenting (to mimic markdown behaviour)
    

    % Basic figure setup, for now with no caption control since it's done
    % automatically by Pandoc (which extracts ![](path) syntax from Markdown).
    \usepackage{graphicx}
    % Keep aspect ratio if custom image width or height is specified
    \setkeys{Gin}{keepaspectratio}
    % Maintain compatibility with old templates. Remove in nbconvert 6.0
    \let\Oldincludegraphics\includegraphics
    % Ensure that by default, figures have no caption (until we provide a
    % proper Figure object with a Caption API and a way to capture that
    % in the conversion process - todo).
    \usepackage{caption}
    \DeclareCaptionFormat{nocaption}{}
    \captionsetup{format=nocaption,aboveskip=0pt,belowskip=0pt}

    \usepackage{float}
    \floatplacement{figure}{H} % forces figures to be placed at the correct location
    \usepackage{xcolor} % Allow colors to be defined
    \usepackage{enumerate} % Needed for markdown enumerations to work
    \usepackage{geometry} % Used to adjust the document margins
    \usepackage{amsmath} % Equations
    \usepackage{amssymb} % Equations
    \usepackage{textcomp} % defines textquotesingle
    % Hack from http://tex.stackexchange.com/a/47451/13684:
    \AtBeginDocument{%
        \def\PYZsq{\textquotesingle}% Upright quotes in Pygmentized code
    }
    \usepackage{upquote} % Upright quotes for verbatim code
    \usepackage{eurosym} % defines \euro

    \usepackage{iftex}
    \ifPDFTeX
        \usepackage[T1]{fontenc}
        \IfFileExists{alphabeta.sty}{
              \usepackage{alphabeta}
          }{
              \usepackage[mathletters]{ucs}
              \usepackage[utf8x]{inputenc}
          }
    \else
        \usepackage{fontspec}
        \usepackage{unicode-math}
    \fi

    \usepackage{fancyvrb} % verbatim replacement that allows latex
    \usepackage{grffile} % extends the file name processing of package graphics
                         % to support a larger range
    \makeatletter % fix for old versions of grffile with XeLaTeX
    \@ifpackagelater{grffile}{2019/11/01}
    {
      % Do nothing on new versions
    }
    {
      \def\Gread@@xetex#1{%
        \IfFileExists{"\Gin@base".bb}%
        {\Gread@eps{\Gin@base.bb}}%
        {\Gread@@xetex@aux#1}%
      }
    }
    \makeatother
    \usepackage[Export]{adjustbox} % Used to constrain images to a maximum size
    \adjustboxset{max size={0.9\linewidth}{0.9\paperheight}}

    % The hyperref package gives us a pdf with properly built
    % internal navigation ('pdf bookmarks' for the table of contents,
    % internal cross-reference links, web links for URLs, etc.)
    \usepackage{hyperref}
    % The default LaTeX title has an obnoxious amount of whitespace. By default,
    % titling removes some of it. It also provides customization options.
    \usepackage{titling}
    \usepackage{longtable} % longtable support required by pandoc >1.10
    \usepackage{booktabs}  % table support for pandoc > 1.12.2
    \usepackage{array}     % table support for pandoc >= 2.11.3
    \usepackage{calc}      % table minipage width calculation for pandoc >= 2.11.1
    \usepackage[inline]{enumitem} % IRkernel/repr support (it uses the enumerate* environment)
    \usepackage[normalem]{ulem} % ulem is needed to support strikethroughs (\sout)
                                % normalem makes italics be italics, not underlines
    \usepackage{soul}      % strikethrough (\st) support for pandoc >= 3.0.0
    \usepackage{mathrsfs}
     % load all other packages
% For cyrillic symbols
\usepackage[english, russian]{babel}


    
    % Colors for the hyperref package
    \definecolor{urlcolor}{rgb}{0,.145,.698}
    \definecolor{linkcolor}{rgb}{.71,0.21,0.01}
    \definecolor{citecolor}{rgb}{.12,.54,.11}

    % ANSI colors
    \definecolor{ansi-black}{HTML}{3E424D}
    \definecolor{ansi-black-intense}{HTML}{282C36}
    \definecolor{ansi-red}{HTML}{E75C58}
    \definecolor{ansi-red-intense}{HTML}{B22B31}
    \definecolor{ansi-green}{HTML}{00A250}
    \definecolor{ansi-green-intense}{HTML}{007427}
    \definecolor{ansi-yellow}{HTML}{DDB62B}
    \definecolor{ansi-yellow-intense}{HTML}{B27D12}
    \definecolor{ansi-blue}{HTML}{208FFB}
    \definecolor{ansi-blue-intense}{HTML}{0065CA}
    \definecolor{ansi-magenta}{HTML}{D160C4}
    \definecolor{ansi-magenta-intense}{HTML}{A03196}
    \definecolor{ansi-cyan}{HTML}{60C6C8}
    \definecolor{ansi-cyan-intense}{HTML}{258F8F}
    \definecolor{ansi-white}{HTML}{C5C1B4}
    \definecolor{ansi-white-intense}{HTML}{A1A6B2}
    \definecolor{ansi-default-inverse-fg}{HTML}{FFFFFF}
    \definecolor{ansi-default-inverse-bg}{HTML}{000000}

    % common color for the border for error outputs.
    \definecolor{outerrorbackground}{HTML}{FFDFDF}

    % commands and environments needed by pandoc snippets
    % extracted from the output of `pandoc -s`
    \providecommand{\tightlist}{%
      \setlength{\itemsep}{0pt}\setlength{\parskip}{0pt}}
    \DefineVerbatimEnvironment{Highlighting}{Verbatim}{commandchars=\\\{\}}
    % Add ',fontsize=\small' for more characters per line
    \newenvironment{Shaded}{}{}
    \newcommand{\KeywordTok}[1]{\textcolor[rgb]{0.00,0.44,0.13}{\textbf{{#1}}}}
    \newcommand{\DataTypeTok}[1]{\textcolor[rgb]{0.56,0.13,0.00}{{#1}}}
    \newcommand{\DecValTok}[1]{\textcolor[rgb]{0.25,0.63,0.44}{{#1}}}
    \newcommand{\BaseNTok}[1]{\textcolor[rgb]{0.25,0.63,0.44}{{#1}}}
    \newcommand{\FloatTok}[1]{\textcolor[rgb]{0.25,0.63,0.44}{{#1}}}
    \newcommand{\CharTok}[1]{\textcolor[rgb]{0.25,0.44,0.63}{{#1}}}
    \newcommand{\StringTok}[1]{\textcolor[rgb]{0.25,0.44,0.63}{{#1}}}
    \newcommand{\CommentTok}[1]{\textcolor[rgb]{0.38,0.63,0.69}{\textit{{#1}}}}
    \newcommand{\OtherTok}[1]{\textcolor[rgb]{0.00,0.44,0.13}{{#1}}}
    \newcommand{\AlertTok}[1]{\textcolor[rgb]{1.00,0.00,0.00}{\textbf{{#1}}}}
    \newcommand{\FunctionTok}[1]{\textcolor[rgb]{0.02,0.16,0.49}{{#1}}}
    \newcommand{\RegionMarkerTok}[1]{{#1}}
    \newcommand{\ErrorTok}[1]{\textcolor[rgb]{1.00,0.00,0.00}{\textbf{{#1}}}}
    \newcommand{\NormalTok}[1]{{#1}}

    % Additional commands for more recent versions of Pandoc
    \newcommand{\ConstantTok}[1]{\textcolor[rgb]{0.53,0.00,0.00}{{#1}}}
    \newcommand{\SpecialCharTok}[1]{\textcolor[rgb]{0.25,0.44,0.63}{{#1}}}
    \newcommand{\VerbatimStringTok}[1]{\textcolor[rgb]{0.25,0.44,0.63}{{#1}}}
    \newcommand{\SpecialStringTok}[1]{\textcolor[rgb]{0.73,0.40,0.53}{{#1}}}
    \newcommand{\ImportTok}[1]{{#1}}
    \newcommand{\DocumentationTok}[1]{\textcolor[rgb]{0.73,0.13,0.13}{\textit{{#1}}}}
    \newcommand{\AnnotationTok}[1]{\textcolor[rgb]{0.38,0.63,0.69}{\textbf{\textit{{#1}}}}}
    \newcommand{\CommentVarTok}[1]{\textcolor[rgb]{0.38,0.63,0.69}{\textbf{\textit{{#1}}}}}
    \newcommand{\VariableTok}[1]{\textcolor[rgb]{0.10,0.09,0.49}{{#1}}}
    \newcommand{\ControlFlowTok}[1]{\textcolor[rgb]{0.00,0.44,0.13}{\textbf{{#1}}}}
    \newcommand{\OperatorTok}[1]{\textcolor[rgb]{0.40,0.40,0.40}{{#1}}}
    \newcommand{\BuiltInTok}[1]{{#1}}
    \newcommand{\ExtensionTok}[1]{{#1}}
    \newcommand{\PreprocessorTok}[1]{\textcolor[rgb]{0.74,0.48,0.00}{{#1}}}
    \newcommand{\AttributeTok}[1]{\textcolor[rgb]{0.49,0.56,0.16}{{#1}}}
    \newcommand{\InformationTok}[1]{\textcolor[rgb]{0.38,0.63,0.69}{\textbf{\textit{{#1}}}}}
    \newcommand{\WarningTok}[1]{\textcolor[rgb]{0.38,0.63,0.69}{\textbf{\textit{{#1}}}}}


    % Define a nice break command that doesn't care if a line doesn't already
    % exist.
    \def\br{\hspace*{\fill} \\* }
    % Math Jax compatibility definitions
    \def\gt{>}
    \def\lt{<}
    \let\Oldtex\TeX
    \let\Oldlatex\LaTeX
    \renewcommand{\TeX}{\textrm{\Oldtex}}
    \renewcommand{\LaTeX}{\textrm{\Oldlatex}}
    % Document parameters
    % Document title
    \title{ad-task-003}
    
    
    
    
    
    
    
% Pygments definitions
\makeatletter
\def\PY@reset{\let\PY@it=\relax \let\PY@bf=\relax%
    \let\PY@ul=\relax \let\PY@tc=\relax%
    \let\PY@bc=\relax \let\PY@ff=\relax}
\def\PY@tok#1{\csname PY@tok@#1\endcsname}
\def\PY@toks#1+{\ifx\relax#1\empty\else%
    \PY@tok{#1}\expandafter\PY@toks\fi}
\def\PY@do#1{\PY@bc{\PY@tc{\PY@ul{%
    \PY@it{\PY@bf{\PY@ff{#1}}}}}}}
\def\PY#1#2{\PY@reset\PY@toks#1+\relax+\PY@do{#2}}

\@namedef{PY@tok@w}{\def\PY@tc##1{\textcolor[rgb]{0.73,0.73,0.73}{##1}}}
\@namedef{PY@tok@c}{\let\PY@it=\textit\def\PY@tc##1{\textcolor[rgb]{0.24,0.48,0.48}{##1}}}
\@namedef{PY@tok@cp}{\def\PY@tc##1{\textcolor[rgb]{0.61,0.40,0.00}{##1}}}
\@namedef{PY@tok@k}{\let\PY@bf=\textbf\def\PY@tc##1{\textcolor[rgb]{0.00,0.50,0.00}{##1}}}
\@namedef{PY@tok@kp}{\def\PY@tc##1{\textcolor[rgb]{0.00,0.50,0.00}{##1}}}
\@namedef{PY@tok@kt}{\def\PY@tc##1{\textcolor[rgb]{0.69,0.00,0.25}{##1}}}
\@namedef{PY@tok@o}{\def\PY@tc##1{\textcolor[rgb]{0.40,0.40,0.40}{##1}}}
\@namedef{PY@tok@ow}{\let\PY@bf=\textbf\def\PY@tc##1{\textcolor[rgb]{0.67,0.13,1.00}{##1}}}
\@namedef{PY@tok@nb}{\def\PY@tc##1{\textcolor[rgb]{0.00,0.50,0.00}{##1}}}
\@namedef{PY@tok@nf}{\def\PY@tc##1{\textcolor[rgb]{0.00,0.00,1.00}{##1}}}
\@namedef{PY@tok@nc}{\let\PY@bf=\textbf\def\PY@tc##1{\textcolor[rgb]{0.00,0.00,1.00}{##1}}}
\@namedef{PY@tok@nn}{\let\PY@bf=\textbf\def\PY@tc##1{\textcolor[rgb]{0.00,0.00,1.00}{##1}}}
\@namedef{PY@tok@ne}{\let\PY@bf=\textbf\def\PY@tc##1{\textcolor[rgb]{0.80,0.25,0.22}{##1}}}
\@namedef{PY@tok@nv}{\def\PY@tc##1{\textcolor[rgb]{0.10,0.09,0.49}{##1}}}
\@namedef{PY@tok@no}{\def\PY@tc##1{\textcolor[rgb]{0.53,0.00,0.00}{##1}}}
\@namedef{PY@tok@nl}{\def\PY@tc##1{\textcolor[rgb]{0.46,0.46,0.00}{##1}}}
\@namedef{PY@tok@ni}{\let\PY@bf=\textbf\def\PY@tc##1{\textcolor[rgb]{0.44,0.44,0.44}{##1}}}
\@namedef{PY@tok@na}{\def\PY@tc##1{\textcolor[rgb]{0.41,0.47,0.13}{##1}}}
\@namedef{PY@tok@nt}{\let\PY@bf=\textbf\def\PY@tc##1{\textcolor[rgb]{0.00,0.50,0.00}{##1}}}
\@namedef{PY@tok@nd}{\def\PY@tc##1{\textcolor[rgb]{0.67,0.13,1.00}{##1}}}
\@namedef{PY@tok@s}{\def\PY@tc##1{\textcolor[rgb]{0.73,0.13,0.13}{##1}}}
\@namedef{PY@tok@sd}{\let\PY@it=\textit\def\PY@tc##1{\textcolor[rgb]{0.73,0.13,0.13}{##1}}}
\@namedef{PY@tok@si}{\let\PY@bf=\textbf\def\PY@tc##1{\textcolor[rgb]{0.64,0.35,0.47}{##1}}}
\@namedef{PY@tok@se}{\let\PY@bf=\textbf\def\PY@tc##1{\textcolor[rgb]{0.67,0.36,0.12}{##1}}}
\@namedef{PY@tok@sr}{\def\PY@tc##1{\textcolor[rgb]{0.64,0.35,0.47}{##1}}}
\@namedef{PY@tok@ss}{\def\PY@tc##1{\textcolor[rgb]{0.10,0.09,0.49}{##1}}}
\@namedef{PY@tok@sx}{\def\PY@tc##1{\textcolor[rgb]{0.00,0.50,0.00}{##1}}}
\@namedef{PY@tok@m}{\def\PY@tc##1{\textcolor[rgb]{0.40,0.40,0.40}{##1}}}
\@namedef{PY@tok@gh}{\let\PY@bf=\textbf\def\PY@tc##1{\textcolor[rgb]{0.00,0.00,0.50}{##1}}}
\@namedef{PY@tok@gu}{\let\PY@bf=\textbf\def\PY@tc##1{\textcolor[rgb]{0.50,0.00,0.50}{##1}}}
\@namedef{PY@tok@gd}{\def\PY@tc##1{\textcolor[rgb]{0.63,0.00,0.00}{##1}}}
\@namedef{PY@tok@gi}{\def\PY@tc##1{\textcolor[rgb]{0.00,0.52,0.00}{##1}}}
\@namedef{PY@tok@gr}{\def\PY@tc##1{\textcolor[rgb]{0.89,0.00,0.00}{##1}}}
\@namedef{PY@tok@ge}{\let\PY@it=\textit}
\@namedef{PY@tok@gs}{\let\PY@bf=\textbf}
\@namedef{PY@tok@ges}{\let\PY@bf=\textbf\let\PY@it=\textit}
\@namedef{PY@tok@gp}{\let\PY@bf=\textbf\def\PY@tc##1{\textcolor[rgb]{0.00,0.00,0.50}{##1}}}
\@namedef{PY@tok@go}{\def\PY@tc##1{\textcolor[rgb]{0.44,0.44,0.44}{##1}}}
\@namedef{PY@tok@gt}{\def\PY@tc##1{\textcolor[rgb]{0.00,0.27,0.87}{##1}}}
\@namedef{PY@tok@err}{\def\PY@bc##1{{\setlength{\fboxsep}{\string -\fboxrule}\fcolorbox[rgb]{1.00,0.00,0.00}{1,1,1}{\strut ##1}}}}
\@namedef{PY@tok@kc}{\let\PY@bf=\textbf\def\PY@tc##1{\textcolor[rgb]{0.00,0.50,0.00}{##1}}}
\@namedef{PY@tok@kd}{\let\PY@bf=\textbf\def\PY@tc##1{\textcolor[rgb]{0.00,0.50,0.00}{##1}}}
\@namedef{PY@tok@kn}{\let\PY@bf=\textbf\def\PY@tc##1{\textcolor[rgb]{0.00,0.50,0.00}{##1}}}
\@namedef{PY@tok@kr}{\let\PY@bf=\textbf\def\PY@tc##1{\textcolor[rgb]{0.00,0.50,0.00}{##1}}}
\@namedef{PY@tok@bp}{\def\PY@tc##1{\textcolor[rgb]{0.00,0.50,0.00}{##1}}}
\@namedef{PY@tok@fm}{\def\PY@tc##1{\textcolor[rgb]{0.00,0.00,1.00}{##1}}}
\@namedef{PY@tok@vc}{\def\PY@tc##1{\textcolor[rgb]{0.10,0.09,0.49}{##1}}}
\@namedef{PY@tok@vg}{\def\PY@tc##1{\textcolor[rgb]{0.10,0.09,0.49}{##1}}}
\@namedef{PY@tok@vi}{\def\PY@tc##1{\textcolor[rgb]{0.10,0.09,0.49}{##1}}}
\@namedef{PY@tok@vm}{\def\PY@tc##1{\textcolor[rgb]{0.10,0.09,0.49}{##1}}}
\@namedef{PY@tok@sa}{\def\PY@tc##1{\textcolor[rgb]{0.73,0.13,0.13}{##1}}}
\@namedef{PY@tok@sb}{\def\PY@tc##1{\textcolor[rgb]{0.73,0.13,0.13}{##1}}}
\@namedef{PY@tok@sc}{\def\PY@tc##1{\textcolor[rgb]{0.73,0.13,0.13}{##1}}}
\@namedef{PY@tok@dl}{\def\PY@tc##1{\textcolor[rgb]{0.73,0.13,0.13}{##1}}}
\@namedef{PY@tok@s2}{\def\PY@tc##1{\textcolor[rgb]{0.73,0.13,0.13}{##1}}}
\@namedef{PY@tok@sh}{\def\PY@tc##1{\textcolor[rgb]{0.73,0.13,0.13}{##1}}}
\@namedef{PY@tok@s1}{\def\PY@tc##1{\textcolor[rgb]{0.73,0.13,0.13}{##1}}}
\@namedef{PY@tok@mb}{\def\PY@tc##1{\textcolor[rgb]{0.40,0.40,0.40}{##1}}}
\@namedef{PY@tok@mf}{\def\PY@tc##1{\textcolor[rgb]{0.40,0.40,0.40}{##1}}}
\@namedef{PY@tok@mh}{\def\PY@tc##1{\textcolor[rgb]{0.40,0.40,0.40}{##1}}}
\@namedef{PY@tok@mi}{\def\PY@tc##1{\textcolor[rgb]{0.40,0.40,0.40}{##1}}}
\@namedef{PY@tok@il}{\def\PY@tc##1{\textcolor[rgb]{0.40,0.40,0.40}{##1}}}
\@namedef{PY@tok@mo}{\def\PY@tc##1{\textcolor[rgb]{0.40,0.40,0.40}{##1}}}
\@namedef{PY@tok@ch}{\let\PY@it=\textit\def\PY@tc##1{\textcolor[rgb]{0.24,0.48,0.48}{##1}}}
\@namedef{PY@tok@cm}{\let\PY@it=\textit\def\PY@tc##1{\textcolor[rgb]{0.24,0.48,0.48}{##1}}}
\@namedef{PY@tok@cpf}{\let\PY@it=\textit\def\PY@tc##1{\textcolor[rgb]{0.24,0.48,0.48}{##1}}}
\@namedef{PY@tok@c1}{\let\PY@it=\textit\def\PY@tc##1{\textcolor[rgb]{0.24,0.48,0.48}{##1}}}
\@namedef{PY@tok@cs}{\let\PY@it=\textit\def\PY@tc##1{\textcolor[rgb]{0.24,0.48,0.48}{##1}}}

\def\PYZbs{\char`\\}
\def\PYZus{\char`\_}
\def\PYZob{\char`\{}
\def\PYZcb{\char`\}}
\def\PYZca{\char`\^}
\def\PYZam{\char`\&}
\def\PYZlt{\char`\<}
\def\PYZgt{\char`\>}
\def\PYZsh{\char`\#}
\def\PYZpc{\char`\%}
\def\PYZdl{\char`\$}
\def\PYZhy{\char`\-}
\def\PYZsq{\char`\'}
\def\PYZdq{\char`\"}
\def\PYZti{\char`\~}
% for compatibility with earlier versions
\def\PYZat{@}
\def\PYZlb{[}
\def\PYZrb{]}
\makeatother


    % For linebreaks inside Verbatim environment from package fancyvrb.
    \makeatletter
        \newbox\Wrappedcontinuationbox
        \newbox\Wrappedvisiblespacebox
        \newcommand*\Wrappedvisiblespace {\textcolor{red}{\textvisiblespace}}
        \newcommand*\Wrappedcontinuationsymbol {\textcolor{red}{\llap{\tiny$\m@th\hookrightarrow$}}}
        \newcommand*\Wrappedcontinuationindent {3ex }
        \newcommand*\Wrappedafterbreak {\kern\Wrappedcontinuationindent\copy\Wrappedcontinuationbox}
        % Take advantage of the already applied Pygments mark-up to insert
        % potential linebreaks for TeX processing.
        %        {, <, #, %, $, ' and ": go to next line.
        %        _, }, ^, &, >, - and ~: stay at end of broken line.
        % Use of \textquotesingle for straight quote.
        \newcommand*\Wrappedbreaksatspecials {%
            \def\PYGZus{\discretionary{\char`\_}{\Wrappedafterbreak}{\char`\_}}%
            \def\PYGZob{\discretionary{}{\Wrappedafterbreak\char`\{}{\char`\{}}%
            \def\PYGZcb{\discretionary{\char`\}}{\Wrappedafterbreak}{\char`\}}}%
            \def\PYGZca{\discretionary{\char`\^}{\Wrappedafterbreak}{\char`\^}}%
            \def\PYGZam{\discretionary{\char`\&}{\Wrappedafterbreak}{\char`\&}}%
            \def\PYGZlt{\discretionary{}{\Wrappedafterbreak\char`\<}{\char`\<}}%
            \def\PYGZgt{\discretionary{\char`\>}{\Wrappedafterbreak}{\char`\>}}%
            \def\PYGZsh{\discretionary{}{\Wrappedafterbreak\char`\#}{\char`\#}}%
            \def\PYGZpc{\discretionary{}{\Wrappedafterbreak\char`\%}{\char`\%}}%
            \def\PYGZdl{\discretionary{}{\Wrappedafterbreak\char`\$}{\char`\$}}%
            \def\PYGZhy{\discretionary{\char`\-}{\Wrappedafterbreak}{\char`\-}}%
            \def\PYGZsq{\discretionary{}{\Wrappedafterbreak\textquotesingle}{\textquotesingle}}%
            \def\PYGZdq{\discretionary{}{\Wrappedafterbreak\char`\"}{\char`\"}}%
            \def\PYGZti{\discretionary{\char`\~}{\Wrappedafterbreak}{\char`\~}}%
        }
        % Some characters . , ; ? ! / are not pygmentized.
        % This macro makes them "active" and they will insert potential linebreaks
        \newcommand*\Wrappedbreaksatpunct {%
            \lccode`\~`\.\lowercase{\def~}{\discretionary{\hbox{\char`\.}}{\Wrappedafterbreak}{\hbox{\char`\.}}}%
            \lccode`\~`\,\lowercase{\def~}{\discretionary{\hbox{\char`\,}}{\Wrappedafterbreak}{\hbox{\char`\,}}}%
            \lccode`\~`\;\lowercase{\def~}{\discretionary{\hbox{\char`\;}}{\Wrappedafterbreak}{\hbox{\char`\;}}}%
            \lccode`\~`\:\lowercase{\def~}{\discretionary{\hbox{\char`\:}}{\Wrappedafterbreak}{\hbox{\char`\:}}}%
            \lccode`\~`\?\lowercase{\def~}{\discretionary{\hbox{\char`\?}}{\Wrappedafterbreak}{\hbox{\char`\?}}}%
            \lccode`\~`\!\lowercase{\def~}{\discretionary{\hbox{\char`\!}}{\Wrappedafterbreak}{\hbox{\char`\!}}}%
            \lccode`\~`\/\lowercase{\def~}{\discretionary{\hbox{\char`\/}}{\Wrappedafterbreak}{\hbox{\char`\/}}}%
            \catcode`\.\active
            \catcode`\,\active
            \catcode`\;\active
            \catcode`\:\active
            \catcode`\?\active
            \catcode`\!\active
            \catcode`\/\active
            \lccode`\~`\~
        }
    \makeatother

    \let\OriginalVerbatim=\Verbatim
    \makeatletter
    \renewcommand{\Verbatim}[1][1]{%
        %\parskip\z@skip
        \sbox\Wrappedcontinuationbox {\Wrappedcontinuationsymbol}%
        \sbox\Wrappedvisiblespacebox {\FV@SetupFont\Wrappedvisiblespace}%
        \def\FancyVerbFormatLine ##1{\hsize\linewidth
            \vtop{\raggedright\hyphenpenalty\z@\exhyphenpenalty\z@
                \doublehyphendemerits\z@\finalhyphendemerits\z@
                \strut ##1\strut}%
        }%
        % If the linebreak is at a space, the latter will be displayed as visible
        % space at end of first line, and a continuation symbol starts next line.
        % Stretch/shrink are however usually zero for typewriter font.
        \def\FV@Space {%
            \nobreak\hskip\z@ plus\fontdimen3\font minus\fontdimen4\font
            \discretionary{\copy\Wrappedvisiblespacebox}{\Wrappedafterbreak}
            {\kern\fontdimen2\font}%
        }%

        % Allow breaks at special characters using \PYG... macros.
        \Wrappedbreaksatspecials
        % Breaks at punctuation characters . , ; ? ! and / need catcode=\active
        \OriginalVerbatim[#1,codes*=\Wrappedbreaksatpunct]%
    }
    \makeatother

    % Exact colors from NB
    \definecolor{incolor}{HTML}{303F9F}
    \definecolor{outcolor}{HTML}{D84315}
    \definecolor{cellborder}{HTML}{CFCFCF}
    \definecolor{cellbackground}{HTML}{F7F7F7}

    % prompt
    \makeatletter
    \newcommand{\boxspacing}{\kern\kvtcb@left@rule\kern\kvtcb@boxsep}
    \makeatother
    \newcommand{\prompt}[4]{
        {\ttfamily\llap{{\color{#2}[#3]:\hspace{3pt}#4}}\vspace{-\baselineskip}}
    }
    

    
    % Prevent overflowing lines due to hard-to-break entities
    \sloppy
    % Setup hyperref package
    \hypersetup{
      breaklinks=true,  % so long urls are correctly broken across lines
      colorlinks=true,
      urlcolor=urlcolor,
      linkcolor=linkcolor,
      citecolor=citecolor,
      }
    % Slightly bigger margins than the latex defaults
    
    \geometry{verbose,tmargin=1in,bmargin=1in,lmargin=1in,rmargin=1in}
    
    

\begin{document}
    
    \maketitle
    
    

    
    \section{Задание 3: Разведочный
анализ}\label{ux437ux430ux434ux430ux43dux438ux435-3-ux440ux430ux437ux432ux435ux434ux43eux447ux43dux44bux439-ux430ux43dux430ux43bux438ux437}

Данные наблюдений за видами (когда и где наблюдается данный вид) типичны
для исследований биоразнообразия. Крупные международные инициативы
поддерживают сбор этих данных волонтерами, например, iNaturalist.
Благодаря таким инициативам, как GBIF, многие из этих данных также
находятся в открытом доступе.

Вы решили поделиться данными полевой кампании, но набор данных все еще
требует некоторой очистки и стандартизации. Например, координаты могут
называться x/y, decimalLatitude/decimalLongitude, lat/long \ldots{} К
счастью, вы знаете о международном стандарте открытых данных для
описания данных о событиях/наблюдениях, т.е. Darwin Core (DwC). Вместо
того, чтобы изобретать собственную модель данных, вы решаете
соответствовать этому международному стандарту. Последнее улучшит
коммуникацию, а также сделает ваши данные совместимыми с GBIF.

Короче говоря, DwC описывает плоскую таблицу (CSV) с согласованным
соглашением об именах для имен заголовков и соглашениями о том, как
должны быть представлены определенные типы данных (для справки,
подробное описание дано здесь). В этом руководстве мы сосредоточимся на
нескольких существующих терминах, чтобы изучить некоторые элементы
очистки данных:

\begin{itemize}
\tightlist
\item
  eventDate : формат дат ISO 6801
\item
  ScientificName : общепринятое научное название вида.
\item
  decimalLatitude / decimalLongitude : координаты вхождения в формате
  WGS84
\item
  sex: либо мужской , либо женский , чтобы охарактеризовать пол события
\item
  instanceID : идентификатор в наборе данных для идентификации отдельных
  записей.
\item
  datasetName : статическая строка, определяющая источник данных
\end{itemize}

Кроме того, дополнительная информация о таксономии будет добавлена с
использованием внешнего API-сервиса.

    \begin{tcolorbox}[breakable, size=fbox, boxrule=1pt, pad at break*=1mm,colback=cellbackground, colframe=cellborder]
\prompt{In}{incolor}{1}{\boxspacing}
\begin{Verbatim}[commandchars=\\\{\}]
\PY{k+kn}{import} \PY{n+nn}{numpy} \PY{k}{as} \PY{n+nn}{np}
\PY{k+kn}{import} \PY{n+nn}{pandas} \PY{k}{as} \PY{n+nn}{pd}
\PY{k+kn}{import} \PY{n+nn}{matplotlib}\PY{n+nn}{.}\PY{n+nn}{pyplot} \PY{k}{as} \PY{n+nn}{plt}

\PY{k+kn}{from} \PY{n+nn}{pyproj} \PY{k+kn}{import} \PY{n}{Transformer}
\PY{k+kn}{from} \PY{n+nn}{calendar} \PY{k+kn}{import} \PY{n}{day\PYZus{}abbr}
\end{Verbatim}
\end{tcolorbox}

    \begin{tcolorbox}[breakable, size=fbox, boxrule=1pt, pad at break*=1mm,colback=cellbackground, colframe=cellborder]
\prompt{In}{incolor}{2}{\boxspacing}
\begin{Verbatim}[commandchars=\\\{\}]
\PY{n}{survey\PYZus{}filename} \PY{o}{=} \PY{l+s+s2}{\PYZdq{}}\PY{l+s+s2}{..}\PY{l+s+se}{\PYZbs{}\PYZbs{}}\PY{l+s+s2}{..}\PY{l+s+se}{\PYZbs{}\PYZbs{}}\PY{l+s+s2}{data}\PY{l+s+se}{\PYZbs{}\PYZbs{}}\PY{l+s+s2}{003}\PY{l+s+se}{\PYZbs{}\PYZbs{}}\PY{l+s+s2}{surveys.csv}\PY{l+s+s2}{\PYZdq{}}
\PY{n}{survey\PYZus{}data} \PY{o}{=} \PY{n}{pd}\PY{o}{.}\PY{n}{read\PYZus{}csv}\PY{p}{(}\PY{n}{survey\PYZus{}filename}\PY{p}{,} \PY{n}{sep}\PY{o}{=}\PY{l+s+s2}{\PYZdq{}}\PY{l+s+s2}{,}\PY{l+s+s2}{\PYZdq{}}\PY{p}{)}
\PY{c+c1}{\PYZsh{}survey\PYZus{}data.head()}
\end{Verbatim}
\end{tcolorbox}

    \begin{tcolorbox}[breakable, size=fbox, boxrule=1pt, pad at break*=1mm,colback=cellbackground, colframe=cellborder]
\prompt{In}{incolor}{3}{\boxspacing}
\begin{Verbatim}[commandchars=\\\{\}]
\PY{l+s+sd}{\PYZsq{}\PYZsq{}\PYZsq{}}
\PY{l+s+sd}{Упражнение 1}
\PY{l+s+sd}{Сколько уникальных записей содержит набор данных?}
\PY{l+s+sd}{\PYZsq{}\PYZsq{}\PYZsq{}}
\PY{c+c1}{\PYZsh{}survey\PYZus{}data.count()}
\PY{n}{survey\PYZus{}data}\PY{o}{.}\PY{n}{info}\PY{p}{(}\PY{p}{)}
\end{Verbatim}
\end{tcolorbox}

    \begin{Verbatim}[commandchars=\\\{\}]
<class 'pandas.core.frame.DataFrame'>
RangeIndex: 35549 entries, 0 to 35548
Data columns (total 8 columns):
 \#   Column     Non-Null Count  Dtype
---  ------     --------------  -----
 0   record\_id  35549 non-null  int64
 1   month      35549 non-null  int64
 2   day        35549 non-null  int64
 3   year       35549 non-null  int64
 4   plot       35549 non-null  int64
 5   species    33534 non-null  object
 6   sex\_char   33042 non-null  object
 7   wgt        32283 non-null  float64
dtypes: float64(1), int64(5), object(2)
memory usage: 2.2+ MB
    \end{Verbatim}

    \begin{tcolorbox}[breakable, size=fbox, boxrule=1pt, pad at break*=1mm,colback=cellbackground, colframe=cellborder]
\prompt{In}{incolor}{4}{\boxspacing}
\begin{Verbatim}[commandchars=\\\{\}]
\PY{l+s+sd}{\PYZsq{}\PYZsq{}\PYZsq{}}
\PY{l+s+sd}{УПРАЖНЕНИЕ 2}
\PY{l+s+sd}{Добавьте новый столбец datasetName в набор данных опроса с datasetname в качестве значения для всех записей (статическое значение для всего набора данных).}
\PY{l+s+sd}{\PYZsq{}\PYZsq{}\PYZsq{}}
\PY{n}{survey\PYZus{}data}\PY{p}{[}\PY{l+s+s2}{\PYZdq{}}\PY{l+s+s2}{datasetname}\PY{l+s+s2}{\PYZdq{}}\PY{p}{]} \PY{o}{=} \PY{l+s+s2}{\PYZdq{}}\PY{l+s+s2}{Ecological Archives E090\PYZhy{}118\PYZhy{}D1.}\PY{l+s+s2}{\PYZdq{}}

\PY{n}{survey\PYZus{}data}\PY{o}{.}\PY{n}{head}\PY{p}{(}\PY{p}{)}
\end{Verbatim}
\end{tcolorbox}

            \begin{tcolorbox}[breakable, size=fbox, boxrule=.5pt, pad at break*=1mm, opacityfill=0]
\prompt{Out}{outcolor}{4}{\boxspacing}
\begin{Verbatim}[commandchars=\\\{\}]
   record\_id  month  day  year  plot species sex\_char  wgt  \textbackslash{}
0          1      7   16  1977     2     NaN        M  NaN
1          2      7   16  1977     3     NaN        M  NaN
2          3      7   16  1977     2      DM        F  NaN
3          4      7   16  1977     7      DM        M  NaN
4          5      7   16  1977     3      DM        M  NaN

                        datasetname
0  Ecological Archives E090-118-D1.
1  Ecological Archives E090-118-D1.
2  Ecological Archives E090-118-D1.
3  Ecological Archives E090-118-D1.
4  Ecological Archives E090-118-D1.
\end{Verbatim}
\end{tcolorbox}
        
    \begin{tcolorbox}[breakable, size=fbox, boxrule=1pt, pad at break*=1mm,colback=cellbackground, colframe=cellborder]
\prompt{In}{incolor}{5}{\boxspacing}
\begin{Verbatim}[commandchars=\\\{\}]
\PY{l+s+sd}{\PYZsq{}\PYZsq{}\PYZsq{}}
\PY{l+s+sd}{УПРАЖНЕНИЕ 3}
\PY{l+s+sd}{Получите список уникальных значений для столбца sex\PYZus{}char.}
\PY{l+s+sd}{\PYZsq{}\PYZsq{}\PYZsq{}}

\PY{n}{survey\PYZus{}data}\PY{p}{[}\PY{l+s+s2}{\PYZdq{}}\PY{l+s+s2}{sex\PYZus{}char}\PY{l+s+s2}{\PYZdq{}}\PY{p}{]}\PY{o}{.}\PY{n}{unique}\PY{p}{(}\PY{p}{)}
\end{Verbatim}
\end{tcolorbox}

            \begin{tcolorbox}[breakable, size=fbox, boxrule=.5pt, pad at break*=1mm, opacityfill=0]
\prompt{Out}{outcolor}{5}{\boxspacing}
\begin{Verbatim}[commandchars=\\\{\}]
array(['M', 'F', nan, 'R', 'P', 'Z'], dtype=object)
\end{Verbatim}
\end{tcolorbox}
        
    \begin{tcolorbox}[breakable, size=fbox, boxrule=1pt, pad at break*=1mm,colback=cellbackground, colframe=cellborder]
\prompt{In}{incolor}{6}{\boxspacing}
\begin{Verbatim}[commandchars=\\\{\}]
\PY{n}{survey\PYZus{}data} \PY{o}{=} \PY{n}{survey\PYZus{}data}\PY{o}{.}\PY{n}{rename}\PY{p}{(}\PY{n}{columns}\PY{o}{=}\PY{p}{\PYZob{}}\PY{l+s+s1}{\PYZsq{}}\PY{l+s+s1}{sex\PYZus{}char}\PY{l+s+s1}{\PYZsq{}}\PY{p}{:} \PY{l+s+s1}{\PYZsq{}}\PY{l+s+s1}{verbatimSex}\PY{l+s+s1}{\PYZsq{}}\PY{p}{\PYZcb{}}\PY{p}{)}
\PY{n}{survey\PYZus{}data}\PY{o}{.}\PY{n}{head}\PY{p}{(}\PY{p}{)}
\end{Verbatim}
\end{tcolorbox}

            \begin{tcolorbox}[breakable, size=fbox, boxrule=.5pt, pad at break*=1mm, opacityfill=0]
\prompt{Out}{outcolor}{6}{\boxspacing}
\begin{Verbatim}[commandchars=\\\{\}]
   record\_id  month  day  year  plot species verbatimSex  wgt  \textbackslash{}
0          1      7   16  1977     2     NaN           M  NaN
1          2      7   16  1977     3     NaN           M  NaN
2          3      7   16  1977     2      DM           F  NaN
3          4      7   16  1977     7      DM           M  NaN
4          5      7   16  1977     3      DM           M  NaN

                        datasetname
0  Ecological Archives E090-118-D1.
1  Ecological Archives E090-118-D1.
2  Ecological Archives E090-118-D1.
3  Ecological Archives E090-118-D1.
4  Ecological Archives E090-118-D1.
\end{Verbatim}
\end{tcolorbox}
        
    \begin{tcolorbox}[breakable, size=fbox, boxrule=1pt, pad at break*=1mm,colback=cellbackground, colframe=cellborder]
\prompt{In}{incolor}{7}{\boxspacing}
\begin{Verbatim}[commandchars=\\\{\}]
\PY{l+s+sd}{\PYZsq{}\PYZsq{}\PYZsq{}}
\PY{l+s+sd}{УПРАЖНЕНИЕ 4}
\PY{l+s+sd}{Выразите отображение значений (например, M \PYZhy{}\PYZgt{} male) в объект словаря Python с именем переменной sex\PYZus{}dict. Значения Z соответствуют Not a Number, который можно определить как np.nan.}
\PY{l+s+sd}{Используйте словарь sex\PYZus{}dict, чтобы заменить значения в столбце verbatimSex новыми значениями и сохранить сопоставленные значения в новом столбце «пол» кадра данных.}
\PY{l+s+sd}{\PYZsq{}\PYZsq{}\PYZsq{}}

\PY{n}{sex\PYZus{}dict} \PY{o}{=} \PY{n+nb}{dict}\PY{p}{(}\PY{p}{\PYZob{}}\PY{l+s+s1}{\PYZsq{}}\PY{l+s+s1}{M}\PY{l+s+s1}{\PYZsq{}}\PY{p}{:} \PY{l+s+s1}{\PYZsq{}}\PY{l+s+s1}{male}\PY{l+s+s1}{\PYZsq{}}\PY{p}{,} \PY{l+s+s1}{\PYZsq{}}\PY{l+s+s1}{F}\PY{l+s+s1}{\PYZsq{}}\PY{p}{:} \PY{l+s+s1}{\PYZsq{}}\PY{l+s+s1}{female}\PY{l+s+s1}{\PYZsq{}}\PY{p}{,} \PY{l+s+s1}{\PYZsq{}}\PY{l+s+s1}{nan}\PY{l+s+s1}{\PYZsq{}}\PY{p}{:} \PY{n}{np}\PY{o}{.}\PY{n}{nan}\PY{p}{,} \PY{l+s+s1}{\PYZsq{}}\PY{l+s+s1}{Z}\PY{l+s+s1}{\PYZsq{}}\PY{p}{:} \PY{n}{np}\PY{o}{.}\PY{n}{nan}\PY{p}{,} \PY{l+s+s1}{\PYZsq{}}\PY{l+s+s1}{R}\PY{l+s+s1}{\PYZsq{}} \PY{p}{:} \PY{l+s+s1}{\PYZsq{}}\PY{l+s+s1}{male}\PY{l+s+s1}{\PYZsq{}}\PY{p}{,} \PY{l+s+s1}{\PYZsq{}}\PY{l+s+s1}{P}\PY{l+s+s1}{\PYZsq{}}\PY{p}{:} \PY{l+s+s1}{\PYZsq{}}\PY{l+s+s1}{female}\PY{l+s+s1}{\PYZsq{}}\PY{p}{\PYZcb{}}\PY{p}{)}
\PY{n}{survey\PYZus{}data}\PY{p}{[}\PY{l+s+s2}{\PYZdq{}}\PY{l+s+s2}{sex}\PY{l+s+s2}{\PYZdq{}}\PY{p}{]} \PY{o}{=} \PY{n}{survey\PYZus{}data}\PY{p}{[}\PY{l+s+s2}{\PYZdq{}}\PY{l+s+s2}{verbatimSex}\PY{l+s+s2}{\PYZdq{}}\PY{p}{]}\PY{o}{.}\PY{n}{map}\PY{p}{(}\PY{n}{sex\PYZus{}dict}\PY{p}{)}

\PY{n}{survey\PYZus{}data}\PY{p}{[}\PY{l+s+s2}{\PYZdq{}}\PY{l+s+s2}{sex}\PY{l+s+s2}{\PYZdq{}}\PY{p}{]}\PY{o}{.}\PY{n}{unique}\PY{p}{(}\PY{p}{)}
\end{Verbatim}
\end{tcolorbox}

            \begin{tcolorbox}[breakable, size=fbox, boxrule=.5pt, pad at break*=1mm, opacityfill=0]
\prompt{Out}{outcolor}{7}{\boxspacing}
\begin{Verbatim}[commandchars=\\\{\}]
array(['male', 'female', nan], dtype=object)
\end{Verbatim}
\end{tcolorbox}
        
    \begin{tcolorbox}[breakable, size=fbox, boxrule=1pt, pad at break*=1mm,colback=cellbackground, colframe=cellborder]
\prompt{In}{incolor}{8}{\boxspacing}
\begin{Verbatim}[commandchars=\\\{\}]
\PY{l+s+sd}{\PYZsq{}\PYZsq{}\PYZsq{}}
\PY{l+s+sd}{УПРАЖНЕНИЕ 5}
\PY{l+s+sd}{Создайте горизонтальную гистограмму, сравнивая количество записей о мужчинах, женщинах и неизвестных (NaN) в наборе данных.}
\PY{l+s+sd}{\PYZsq{}\PYZsq{}\PYZsq{}}

\PY{n}{grouped1} \PY{o}{=} \PY{n}{survey\PYZus{}data}\PY{o}{.}\PY{n}{groupby}\PY{p}{(}\PY{l+s+s2}{\PYZdq{}}\PY{l+s+s2}{sex}\PY{l+s+s2}{\PYZdq{}}\PY{p}{,} \PY{n}{dropna}\PY{o}{=}\PY{k+kc}{False}\PY{p}{)}\PY{o}{.}\PY{n}{count}\PY{p}{(}\PY{p}{)}\PY{p}{[}\PY{l+s+s2}{\PYZdq{}}\PY{l+s+s2}{record\PYZus{}id}\PY{l+s+s2}{\PYZdq{}}\PY{p}{]}
\PY{c+c1}{\PYZsh{}grouped1.head()}
\PY{n}{grouped1}\PY{o}{.}\PY{n}{plot}\PY{o}{.}\PY{n}{barh}\PY{p}{(}\PY{n}{grid}\PY{o}{=}\PY{k+kc}{True}\PY{p}{,} \PY{n}{legend}\PY{o}{=}\PY{k+kc}{False}\PY{p}{)}
\PY{n}{plt}\PY{o}{.}\PY{n}{show}\PY{p}{(}\PY{p}{)}
\end{Verbatim}
\end{tcolorbox}

    \begin{center}
    \adjustimage{max size={0.9\linewidth}{0.9\paperheight}}{output_8_0.png}
    \end{center}
    { \hspace*{\fill} \\}
    
    \begin{tcolorbox}[breakable, size=fbox, boxrule=1pt, pad at break*=1mm,colback=cellbackground, colframe=cellborder]
\prompt{In}{incolor}{9}{\boxspacing}
\begin{Verbatim}[commandchars=\\\{\}]
\PY{n}{survey\PYZus{}data}\PY{p}{[}\PY{l+s+s2}{\PYZdq{}}\PY{l+s+s2}{species}\PY{l+s+s2}{\PYZdq{}}\PY{p}{]}\PY{o}{.}\PY{n}{unique}\PY{p}{(}\PY{p}{)}
\PY{c+c1}{\PYZsh{}survey\PYZus{}data.head(10)}
\end{Verbatim}
\end{tcolorbox}

            \begin{tcolorbox}[breakable, size=fbox, boxrule=.5pt, pad at break*=1mm, opacityfill=0]
\prompt{Out}{outcolor}{9}{\boxspacing}
\begin{Verbatim}[commandchars=\\\{\}]
array([nan, 'DM', 'PF', 'PE', 'DM and SH', 'DS', 'PP', 'SH', 'OT', 'DO',
       'OX', 'SS', 'OL', 'RM', 'SA', 'PM', 'AH', 'DX', 'AB', 'CB', 'CM',
       'CQ', 'RF', 'PC', 'PG', 'PH', 'PU', 'CV', 'UR', 'UP', 'ZL', 'UL',
       'CS', 'SC', 'BA', 'SF', 'RO', 'AS', 'SO', 'PI', 'ST', 'CU', 'SU',
       'RX', 'PB', 'PL', 'PX', 'CT', 'US'], dtype=object)
\end{Verbatim}
\end{tcolorbox}
        
    \begin{tcolorbox}[breakable, size=fbox, boxrule=1pt, pad at break*=1mm,colback=cellbackground, colframe=cellborder]
\prompt{In}{incolor}{10}{\boxspacing}
\begin{Verbatim}[commandchars=\\\{\}]
\PY{k}{def} \PY{n+nf}{solve\PYZus{}double\PYZus{}field\PYZus{}entry}\PY{p}{(}\PY{n}{df}\PY{p}{,} \PY{n}{keyword}\PY{o}{=}\PY{l+s+s2}{\PYZdq{}}\PY{l+s+s2}{and}\PY{l+s+s2}{\PYZdq{}}\PY{p}{,} \PY{n}{column}\PY{o}{=}\PY{l+s+s2}{\PYZdq{}}\PY{l+s+s2}{verbatimEventDate}\PY{l+s+s2}{\PYZdq{}}\PY{p}{)}\PY{p}{:}
\PY{+w}{    }\PY{l+s+sd}{\PYZdq{}\PYZdq{}\PYZdq{} Разделить по ключевому слову в столбце для перечисления и создать дополнительную запись}

\PY{l+s+sd}{    Параметры}
\PY{l+s+sd}{    \PYZhy{}\PYZhy{}\PYZhy{}\PYZhy{}\PYZhy{}\PYZhy{}\PYZhy{}\PYZhy{}\PYZhy{}\PYZhy{}}
\PY{l+s+sd}{    df: pd.DataFrame}
\PY{l+s+sd}{    DataFrame с двойной записью поля в одном или нескольких значениях}
\PY{l+s+sd}{    keyword: str}
\PY{l+s+sd}{    слово/символ для разделения двойные записи в}
\PY{l+s+sd}{    column: имя столбца str}
\PY{l+s+sd}{    для разделения записей}
\PY{l+s+sd}{    \PYZdq{}\PYZdq{}\PYZdq{}}
    
    \PY{n}{df} \PY{o}{=} \PY{n}{df}\PY{o}{.}\PY{n}{copy}\PY{p}{(}\PY{p}{)} \PY{c+c1}{\PYZsh{} copy the input DataFrame to avoid editing the original}
    \PY{n}{df}\PY{p}{[}\PY{n}{column}\PY{p}{]} \PY{o}{=} \PY{n}{df}\PY{p}{[}\PY{n}{column}\PY{p}{]}\PY{o}{.}\PY{n}{str}\PY{o}{.}\PY{n}{split}\PY{p}{(}\PY{n}{keyword}\PY{p}{)}
    \PY{n}{df} \PY{o}{=} \PY{n}{df}\PY{o}{.}\PY{n}{explode}\PY{p}{(}\PY{n}{column}\PY{p}{)}
    \PY{n}{df}\PY{p}{[}\PY{n}{column}\PY{p}{]} \PY{o}{=} \PY{n}{df}\PY{p}{[}\PY{n}{column}\PY{p}{]}\PY{o}{.}\PY{n}{str}\PY{o}{.}\PY{n}{strip}\PY{p}{(}\PY{p}{)} \PY{c+c1}{\PYZsh{} remove white space around the words}
    \PY{k}{return} \PY{n}{df}
\end{Verbatim}
\end{tcolorbox}

    \begin{tcolorbox}[breakable, size=fbox, boxrule=1pt, pad at break*=1mm,colback=cellbackground, colframe=cellborder]
\prompt{In}{incolor}{11}{\boxspacing}
\begin{Verbatim}[commandchars=\\\{\}]
\PY{l+s+sd}{\PYZsq{}\PYZsq{}\PYZsq{}}
\PY{l+s+sd}{УПРАЖНЕНИЕ 6}
\PY{l+s+sd}{Используйте функцию solve\PYZus{}double\PYZus{}field\PYZus{}entry, чтобы обновить данные обзора, отделив двойные записи. }
\PY{l+s+sd}{Сохраните результат как переменную survey\PYZus{}data\PYZus{}decoupled.}
\PY{l+s+sd}{\PYZsq{}\PYZsq{}\PYZsq{}}

\PY{n}{survey\PYZus{}data\PYZus{}decoupled} \PY{o}{=} \PY{n}{solve\PYZus{}double\PYZus{}field\PYZus{}entry}\PY{p}{(}\PY{n}{survey\PYZus{}data}\PY{p}{,} \PY{n}{keyword} \PY{o}{=} \PY{l+s+s2}{\PYZdq{}}\PY{l+s+s2}{and}\PY{l+s+s2}{\PYZdq{}}\PY{p}{,} \PY{n}{column} \PY{o}{=} \PY{l+s+s2}{\PYZdq{}}\PY{l+s+s2}{species}\PY{l+s+s2}{\PYZdq{}}\PY{p}{)}
\PY{c+c1}{\PYZsh{}survey\PYZus{}data\PYZus{}decoupled[\PYZdq{}species\PYZdq{}].unique()}
\PY{n}{survey\PYZus{}data\PYZus{}decoupled}\PY{o}{.}\PY{n}{head}\PY{p}{(}\PY{l+m+mi}{10}\PY{p}{)}
\end{Verbatim}
\end{tcolorbox}

            \begin{tcolorbox}[breakable, size=fbox, boxrule=.5pt, pad at break*=1mm, opacityfill=0]
\prompt{Out}{outcolor}{11}{\boxspacing}
\begin{Verbatim}[commandchars=\\\{\}]
   record\_id  month  day  year  plot species verbatimSex  wgt  \textbackslash{}
0          1      7   16  1977     2     NaN           M  NaN
1          2      7   16  1977     3     NaN           M  NaN
2          3      7   16  1977     2      DM           F  NaN
3          4      7   16  1977     7      DM           M  NaN
4          5      7   16  1977     3      DM           M  NaN
5          6      7   16  1977     1      PF           M  NaN
6          7      7   16  1977     2      PE           F  NaN
7          8      7   16  1977     1      DM           M  NaN
8          9      7   16  1977     1      DM         NaN  NaN
8          9      7   16  1977     1      SH         NaN  NaN

                        datasetname     sex
0  Ecological Archives E090-118-D1.    male
1  Ecological Archives E090-118-D1.    male
2  Ecological Archives E090-118-D1.  female
3  Ecological Archives E090-118-D1.    male
4  Ecological Archives E090-118-D1.    male
5  Ecological Archives E090-118-D1.    male
6  Ecological Archives E090-118-D1.  female
7  Ecological Archives E090-118-D1.    male
8  Ecological Archives E090-118-D1.     NaN
8  Ecological Archives E090-118-D1.     NaN
\end{Verbatim}
\end{tcolorbox}
        
    \begin{tcolorbox}[breakable, size=fbox, boxrule=1pt, pad at break*=1mm,colback=cellbackground, colframe=cellborder]
\prompt{In}{incolor}{12}{\boxspacing}
\begin{Verbatim}[commandchars=\\\{\}]
\PY{n}{survey\PYZus{}data\PYZus{}decoupled}\PY{p}{[}\PY{l+s+s2}{\PYZdq{}}\PY{l+s+s2}{occurrence\PYZus{}id}\PY{l+s+s2}{\PYZdq{}}\PY{p}{]} \PY{o}{=} \PY{n}{np}\PY{o}{.}\PY{n}{arange}\PY{p}{(}\PY{l+m+mi}{1}\PY{p}{,} \PY{n+nb}{len}\PY{p}{(}\PY{n}{survey\PYZus{}data\PYZus{}decoupled}\PY{p}{)} \PY{o}{+} \PY{l+m+mi}{1}\PY{p}{,} \PY{l+m+mi}{1}\PY{p}{)}
\PY{n}{survey\PYZus{}data\PYZus{}decoupled} \PY{o}{=} \PY{n}{survey\PYZus{}data\PYZus{}decoupled}\PY{o}{.}\PY{n}{drop}\PY{p}{(}\PY{n}{columns}\PY{o}{=}\PY{l+s+s2}{\PYZdq{}}\PY{l+s+s2}{record\PYZus{}id}\PY{l+s+s2}{\PYZdq{}}\PY{p}{)}
\PY{n}{survey\PYZus{}data\PYZus{}decoupled}\PY{o}{.}\PY{n}{head}\PY{p}{(}\PY{l+m+mi}{10}\PY{p}{)}
\end{Verbatim}
\end{tcolorbox}

            \begin{tcolorbox}[breakable, size=fbox, boxrule=.5pt, pad at break*=1mm, opacityfill=0]
\prompt{Out}{outcolor}{12}{\boxspacing}
\begin{Verbatim}[commandchars=\\\{\}]
   month  day  year  plot species verbatimSex  wgt  \textbackslash{}
0      7   16  1977     2     NaN           M  NaN
1      7   16  1977     3     NaN           M  NaN
2      7   16  1977     2      DM           F  NaN
3      7   16  1977     7      DM           M  NaN
4      7   16  1977     3      DM           M  NaN
5      7   16  1977     1      PF           M  NaN
6      7   16  1977     2      PE           F  NaN
7      7   16  1977     1      DM           M  NaN
8      7   16  1977     1      DM         NaN  NaN
8      7   16  1977     1      SH         NaN  NaN

                        datasetname     sex  occurrence\_id
0  Ecological Archives E090-118-D1.    male              1
1  Ecological Archives E090-118-D1.    male              2
2  Ecological Archives E090-118-D1.  female              3
3  Ecological Archives E090-118-D1.    male              4
4  Ecological Archives E090-118-D1.    male              5
5  Ecological Archives E090-118-D1.    male              6
6  Ecological Archives E090-118-D1.  female              7
7  Ecological Archives E090-118-D1.    male              8
8  Ecological Archives E090-118-D1.     NaN              9
8  Ecological Archives E090-118-D1.     NaN             10
\end{Verbatim}
\end{tcolorbox}
        
    \begin{tcolorbox}[breakable, size=fbox, boxrule=1pt, pad at break*=1mm,colback=cellbackground, colframe=cellborder]
\prompt{In}{incolor}{13}{\boxspacing}
\begin{Verbatim}[commandchars=\\\{\}]
\PY{l+s+sd}{\PYZsq{}\PYZsq{}\PYZsq{}}
\PY{l+s+sd}{УПРАЖНЕНИЕ 7}

\PY{l+s+sd}{Сделайте выборку survey\PYZus{}data\PYZus{}decoupled, содержащую те записи, которые не могут быть правильно интерпретированы как значения даты, и сохраните полученный DataFrame как новую переменную trouble\PYZus{}makers.}
\PY{l+s+sd}{\PYZsq{}\PYZsq{}\PYZsq{}}

\PY{n}{markers} \PY{o}{=} \PY{n}{pd}\PY{o}{.}\PY{n}{to\PYZus{}datetime}\PY{p}{(}\PY{n}{survey\PYZus{}data\PYZus{}decoupled}\PY{p}{[}\PY{p}{[}\PY{l+s+s2}{\PYZdq{}}\PY{l+s+s2}{year}\PY{l+s+s2}{\PYZdq{}}\PY{p}{,} \PY{l+s+s2}{\PYZdq{}}\PY{l+s+s2}{month}\PY{l+s+s2}{\PYZdq{}}\PY{p}{,} \PY{l+s+s2}{\PYZdq{}}\PY{l+s+s2}{day}\PY{l+s+s2}{\PYZdq{}}\PY{p}{]}\PY{p}{]}\PY{p}{,} \PY{n}{errors}\PY{o}{=}\PY{l+s+s1}{\PYZsq{}}\PY{l+s+s1}{coerce}\PY{l+s+s1}{\PYZsq{}}\PY{p}{)}\PY{o}{.}\PY{n}{isna}\PY{p}{(}\PY{p}{)}
\PY{n}{trouble\PYZus{}makers} \PY{o}{=} \PY{n}{survey\PYZus{}data\PYZus{}decoupled}\PY{p}{[}\PY{n}{markers}\PY{p}{]}

\PY{n}{trouble\PYZus{}makers}\PY{o}{.}\PY{n}{head}\PY{p}{(}\PY{p}{)}
\PY{c+c1}{\PYZsh{}trouble\PYZus{}makers[\PYZdq{}day\PYZdq{}].unique()}
\PY{c+c1}{\PYZsh{}trouble\PYZus{}makers[\PYZdq{}month\PYZdq{}].unique()}
\PY{c+c1}{\PYZsh{}trouble\PYZus{}makers[\PYZdq{}year\PYZdq{}].unique()}
\end{Verbatim}
\end{tcolorbox}

            \begin{tcolorbox}[breakable, size=fbox, boxrule=.5pt, pad at break*=1mm, opacityfill=0]
\prompt{Out}{outcolor}{13}{\boxspacing}
\begin{Verbatim}[commandchars=\\\{\}]
       month  day  year  plot species verbatimSex   wgt  \textbackslash{}
30649      4   31  2000     6      PP           F  19.0
30650      4   31  2000     6      PB           M  32.0
30651      4   31  2000     6      PB           F  30.0
30652      4   31  2000     6      PP           M  20.0
30653      4   31  2000     6      PP           M  24.0

                            datasetname     sex  occurrence\_id
30649  Ecological Archives E090-118-D1.  female          30651
30650  Ecological Archives E090-118-D1.    male          30652
30651  Ecological Archives E090-118-D1.  female          30653
30652  Ecological Archives E090-118-D1.    male          30654
30653  Ecological Archives E090-118-D1.    male          30655
\end{Verbatim}
\end{tcolorbox}
        
    \begin{tcolorbox}[breakable, size=fbox, boxrule=1pt, pad at break*=1mm,colback=cellbackground, colframe=cellborder]
\prompt{In}{incolor}{14}{\boxspacing}
\begin{Verbatim}[commandchars=\\\{\}]
\PY{l+s+sd}{\PYZsq{}\PYZsq{}\PYZsq{}}
\PY{l+s+sd}{\PYZsh{}\PYZsh{}\PYZsh{}\PYZsh{} УПРАЖНЕНИЕ 8}
\PY{l+s+sd}{Присвойте во фрейме данных survey\PYZus{}data\PYZus{}decoupled всем значениям дня нарушителей спокойствия значение 30 вместо 31.}
\PY{l+s+sd}{\PYZsq{}\PYZsq{}\PYZsq{}}
\PY{n}{survey\PYZus{}data\PYZus{}decoupled}\PY{o}{.}\PY{n}{loc}\PY{p}{[}\PY{n}{markers}\PY{p}{,} \PY{p}{[}\PY{l+s+s2}{\PYZdq{}}\PY{l+s+s2}{day}\PY{l+s+s2}{\PYZdq{}}\PY{p}{]}\PY{p}{]} \PY{o}{=} \PY{l+m+mi}{30}

\PY{c+c1}{\PYZsh{} check}
\PY{n}{pd}\PY{o}{.}\PY{n}{to\PYZus{}datetime}\PY{p}{(}\PY{n}{survey\PYZus{}data\PYZus{}decoupled}\PY{p}{[}\PY{p}{[}\PY{l+s+s2}{\PYZdq{}}\PY{l+s+s2}{year}\PY{l+s+s2}{\PYZdq{}}\PY{p}{,} \PY{l+s+s2}{\PYZdq{}}\PY{l+s+s2}{month}\PY{l+s+s2}{\PYZdq{}}\PY{p}{,} \PY{l+s+s2}{\PYZdq{}}\PY{l+s+s2}{day}\PY{l+s+s2}{\PYZdq{}}\PY{p}{]}\PY{p}{]}\PY{p}{,} \PY{n}{errors}\PY{o}{=}\PY{l+s+s1}{\PYZsq{}}\PY{l+s+s1}{coerce}\PY{l+s+s1}{\PYZsq{}}\PY{p}{)}\PY{o}{.}\PY{n}{isna}\PY{p}{(}\PY{p}{)}\PY{o}{.}\PY{n}{sum}\PY{p}{(}\PY{p}{)}
\end{Verbatim}
\end{tcolorbox}

            \begin{tcolorbox}[breakable, size=fbox, boxrule=.5pt, pad at break*=1mm, opacityfill=0]
\prompt{Out}{outcolor}{14}{\boxspacing}
\begin{Verbatim}[commandchars=\\\{\}]
0
\end{Verbatim}
\end{tcolorbox}
        
    \begin{tcolorbox}[breakable, size=fbox, boxrule=1pt, pad at break*=1mm,colback=cellbackground, colframe=cellborder]
\prompt{In}{incolor}{15}{\boxspacing}
\begin{Verbatim}[commandchars=\\\{\}]
\PY{n}{survey\PYZus{}data\PYZus{}decoupled}\PY{p}{[}\PY{l+s+s2}{\PYZdq{}}\PY{l+s+s2}{eventDate}\PY{l+s+s2}{\PYZdq{}}\PY{p}{]} \PY{o}{=} \PY{n}{pd}\PY{o}{.}\PY{n}{to\PYZus{}datetime}\PY{p}{(}\PY{n}{survey\PYZus{}data\PYZus{}decoupled}\PY{p}{[}\PY{p}{[}\PY{l+s+s2}{\PYZdq{}}\PY{l+s+s2}{year}\PY{l+s+s2}{\PYZdq{}}\PY{p}{,} \PY{l+s+s2}{\PYZdq{}}\PY{l+s+s2}{month}\PY{l+s+s2}{\PYZdq{}}\PY{p}{,} \PY{l+s+s2}{\PYZdq{}}\PY{l+s+s2}{day}\PY{l+s+s2}{\PYZdq{}}\PY{p}{]}\PY{p}{]}\PY{p}{)}
\PY{n}{survey\PYZus{}data\PYZus{}decoupled} \PY{o}{=} \PY{n}{survey\PYZus{}data\PYZus{}decoupled}\PY{o}{.}\PY{n}{drop}\PY{p}{(}\PY{n}{columns}\PY{o}{=}\PY{p}{[}\PY{l+s+s2}{\PYZdq{}}\PY{l+s+s2}{day}\PY{l+s+s2}{\PYZdq{}}\PY{p}{,} \PY{l+s+s2}{\PYZdq{}}\PY{l+s+s2}{month}\PY{l+s+s2}{\PYZdq{}}\PY{p}{,} \PY{l+s+s2}{\PYZdq{}}\PY{l+s+s2}{year}\PY{l+s+s2}{\PYZdq{}}\PY{p}{]}\PY{p}{)}

\PY{n}{survey\PYZus{}data\PYZus{}decoupled}\PY{p}{[} \PY{l+s+s2}{\PYZdq{}}\PY{l+s+s2}{eventDate}\PY{l+s+s2}{\PYZdq{}} \PY{p}{]}\PY{o}{.}\PY{n}{dtype}
\end{Verbatim}
\end{tcolorbox}

            \begin{tcolorbox}[breakable, size=fbox, boxrule=.5pt, pad at break*=1mm, opacityfill=0]
\prompt{Out}{outcolor}{15}{\boxspacing}
\begin{Verbatim}[commandchars=\\\{\}]
dtype('<M8[ns]')
\end{Verbatim}
\end{tcolorbox}
        
    \begin{tcolorbox}[breakable, size=fbox, boxrule=1pt, pad at break*=1mm,colback=cellbackground, colframe=cellborder]
\prompt{In}{incolor}{16}{\boxspacing}
\begin{Verbatim}[commandchars=\\\{\}]
\PY{l+s+sd}{\PYZsq{}\PYZsq{}\PYZsq{}}
\PY{l+s+sd}{УПРАЖНЕНИЕ 10}

\PY{l+s+sd}{Создайте горизонтальную гистограмму с количеством записей для каждого года, }
\PY{l+s+sd}{но без использования столбца года, используя непосредственно столбец eventDate.}
\PY{l+s+sd}{\PYZsq{}\PYZsq{}\PYZsq{}}

\PY{n}{grouped1} \PY{o}{=} \PY{n}{survey\PYZus{}data\PYZus{}decoupled}\PY{o}{.}\PY{n}{groupby}\PY{p}{(}\PY{n}{survey\PYZus{}data\PYZus{}decoupled}\PY{p}{[}\PY{l+s+s2}{\PYZdq{}}\PY{l+s+s2}{eventDate}\PY{l+s+s2}{\PYZdq{}}\PY{p}{]}\PY{o}{.}\PY{n}{dt}\PY{o}{.}\PY{n}{year}\PY{p}{,} \PY{n}{dropna}\PY{o}{=}\PY{k+kc}{False}\PY{p}{)}\PY{o}{.}\PY{n}{count}\PY{p}{(}\PY{p}{)}\PY{p}{[}\PY{l+s+s2}{\PYZdq{}}\PY{l+s+s2}{occurrence\PYZus{}id}\PY{l+s+s2}{\PYZdq{}}\PY{p}{]}
\PY{c+c1}{\PYZsh{}grouped1.head()}
\PY{n}{grouped1}\PY{o}{.}\PY{n}{plot}\PY{o}{.}\PY{n}{barh}\PY{p}{(}\PY{n}{grid}\PY{o}{=}\PY{k+kc}{True}\PY{p}{,} \PY{n}{legend}\PY{o}{=}\PY{k+kc}{False}\PY{p}{,} \PY{n}{title}\PY{o}{=}\PY{l+s+s2}{\PYZdq{}}\PY{l+s+s2}{Количество записей по годам}\PY{l+s+s2}{\PYZdq{}}\PY{p}{)}
\PY{n}{plt}\PY{o}{.}\PY{n}{show}\PY{p}{(}\PY{p}{)}
\end{Verbatim}
\end{tcolorbox}

    \begin{center}
    \adjustimage{max size={0.9\linewidth}{0.9\paperheight}}{output_16_0.png}
    \end{center}
    { \hspace*{\fill} \\}
    
    \begin{tcolorbox}[breakable, size=fbox, boxrule=1pt, pad at break*=1mm,colback=cellbackground, colframe=cellborder]
\prompt{In}{incolor}{17}{\boxspacing}
\begin{Verbatim}[commandchars=\\\{\}]
\PY{l+s+sd}{\PYZsq{}\PYZsq{}\PYZsq{}}
\PY{l+s+sd}{УПРАЖНЕНИЕ 11}
\PY{l+s+sd}{Создайте гистограмму с количеством записей для каждого дня недели (dayofweek)}
\PY{l+s+sd}{\PYZsq{}\PYZsq{}\PYZsq{}}

\PY{n}{grouped1} \PY{o}{=} \PY{n}{survey\PYZus{}data\PYZus{}decoupled}\PY{o}{.}\PY{n}{groupby}\PY{p}{(}\PY{n}{survey\PYZus{}data\PYZus{}decoupled}\PY{p}{[}\PY{l+s+s2}{\PYZdq{}}\PY{l+s+s2}{eventDate}\PY{l+s+s2}{\PYZdq{}}\PY{p}{]}\PY{o}{.}\PY{n}{dt}\PY{o}{.}\PY{n}{dayofweek}\PY{p}{,} \PY{n}{dropna}\PY{o}{=}\PY{k+kc}{False}\PY{p}{)}\PY{o}{.}\PY{n}{count}\PY{p}{(}\PY{p}{)}\PY{p}{[}\PY{l+s+s2}{\PYZdq{}}\PY{l+s+s2}{occurrence\PYZus{}id}\PY{l+s+s2}{\PYZdq{}}\PY{p}{]}
\PY{c+c1}{\PYZsh{}grouped1.head()}

\PY{n}{ax} \PY{o}{=} \PY{n}{grouped1}\PY{o}{.}\PY{n}{plot}\PY{o}{.}\PY{n}{barh}\PY{p}{(}\PY{n}{grid}\PY{o}{=}\PY{k+kc}{True}\PY{p}{,} \PY{n}{legend}\PY{o}{=}\PY{k+kc}{False}\PY{p}{,} \PY{n}{title}\PY{o}{=}\PY{l+s+s2}{\PYZdq{}}\PY{l+s+s2}{Количество записей по дням недели}\PY{l+s+s2}{\PYZdq{}}\PY{p}{)}

\PY{n}{ticks\PYZus{}loc} \PY{o}{=} \PY{n}{ax}\PY{o}{.}\PY{n}{get\PYZus{}yticks}\PY{p}{(}\PY{p}{)}\PY{o}{.}\PY{n}{tolist}\PY{p}{(}\PY{p}{)}
\PY{n}{ax}\PY{o}{.}\PY{n}{yaxis}\PY{o}{.}\PY{n}{set\PYZus{}ticks}\PY{p}{(}\PY{n}{ticks\PYZus{}loc}\PY{p}{)}
\PY{n}{values\PYZus{}list} \PY{o}{=} \PY{n+nb}{list}\PY{p}{(}\PY{n}{day\PYZus{}abbr}\PY{p}{)}
\PY{n}{ax}\PY{o}{.}\PY{n}{set\PYZus{}yticklabels}\PY{p}{(}\PY{n}{values\PYZus{}list}\PY{p}{)}
\PY{n}{ax}\PY{o}{.}\PY{n}{set\PYZus{}ylabel}\PY{p}{(}\PY{l+s+s2}{\PYZdq{}}\PY{l+s+s2}{день недели}\PY{l+s+s2}{\PYZdq{}}\PY{p}{)} 

\PY{n}{plt}\PY{o}{.}\PY{n}{show}\PY{p}{(}\PY{p}{)}
\end{Verbatim}
\end{tcolorbox}

    \begin{center}
    \adjustimage{max size={0.9\linewidth}{0.9\paperheight}}{output_17_0.png}
    \end{center}
    { \hspace*{\fill} \\}
    
    \begin{tcolorbox}[breakable, size=fbox, boxrule=1pt, pad at break*=1mm,colback=cellbackground, colframe=cellborder]
\prompt{In}{incolor}{18}{\boxspacing}
\begin{Verbatim}[commandchars=\\\{\}]
\PY{l+s+sd}{\PYZsq{}\PYZsq{}\PYZsq{}}
\PY{l+s+sd}{УПРАЖНЕНИЕ 12}
\PY{l+s+sd}{Прочитайте файл «species.csv» и сохраните полученный набор данных как переменную species\PYZus{}data.}
\PY{l+s+sd}{\PYZsq{}\PYZsq{}\PYZsq{}}
\PY{n}{spec\PYZus{}filename} \PY{o}{=} \PY{l+s+s2}{\PYZdq{}}\PY{l+s+s2}{..}\PY{l+s+se}{\PYZbs{}\PYZbs{}}\PY{l+s+s2}{..}\PY{l+s+se}{\PYZbs{}\PYZbs{}}\PY{l+s+s2}{data}\PY{l+s+se}{\PYZbs{}\PYZbs{}}\PY{l+s+s2}{003}\PY{l+s+se}{\PYZbs{}\PYZbs{}}\PY{l+s+s2}{species.csv}\PY{l+s+s2}{\PYZdq{}}
\PY{n}{species\PYZus{}data} \PY{o}{=} \PY{n}{pd}\PY{o}{.}\PY{n}{read\PYZus{}csv}\PY{p}{(}\PY{n}{spec\PYZus{}filename}\PY{p}{,} \PY{n}{sep}\PY{o}{=}\PY{l+s+s2}{\PYZdq{}}\PY{l+s+s2}{;}\PY{l+s+s2}{\PYZdq{}}\PY{p}{)}
\PY{n}{species\PYZus{}data}\PY{o}{.}\PY{n}{head}\PY{p}{(}\PY{p}{)}
\end{Verbatim}
\end{tcolorbox}

            \begin{tcolorbox}[breakable, size=fbox, boxrule=.5pt, pad at break*=1mm, opacityfill=0]
\prompt{Out}{outcolor}{18}{\boxspacing}
\begin{Verbatim}[commandchars=\\\{\}]
  species\_id             genus          species                 taxa
0         AB        Amphispiza        bilineata                 Bird
1         AH  Ammospermophilus          harrisi  Rodent-not censused
2         AS        Ammodramus       savannarum                 Bird
3         BA           Baiomys          taylori               Rodent
4         CB   Campylorhynchus  brunneicapillus                 Bird
\end{Verbatim}
\end{tcolorbox}
        
    \begin{tcolorbox}[breakable, size=fbox, boxrule=1pt, pad at break*=1mm,colback=cellbackground, colframe=cellborder]
\prompt{In}{incolor}{19}{\boxspacing}
\begin{Verbatim}[commandchars=\\\{\}]
\PY{l+s+sd}{\PYZsq{}\PYZsq{}\PYZsq{}}
\PY{l+s+sd}{УПРАЖНЕНИЕ 13}
\PY{l+s+sd}{Преобразуйте значение «NE» в «NA», используя логическое индексирование/фильтрацию для столбца species\PYZus{}id.}
\PY{l+s+sd}{\PYZsq{}\PYZsq{}\PYZsq{}}
\PY{n}{species\PYZus{}data}\PY{o}{.}\PY{n}{loc}\PY{p}{[}\PY{n}{species\PYZus{}data}\PY{p}{[}\PY{l+s+s2}{\PYZdq{}}\PY{l+s+s2}{species\PYZus{}id}\PY{l+s+s2}{\PYZdq{}}\PY{p}{]} \PY{o}{==} \PY{l+s+s2}{\PYZdq{}}\PY{l+s+s2}{NE}\PY{l+s+s2}{\PYZdq{}}\PY{p}{,} \PY{p}{[}\PY{l+s+s2}{\PYZdq{}}\PY{l+s+s2}{species\PYZus{}id}\PY{l+s+s2}{\PYZdq{}}\PY{p}{]}\PY{p}{]} \PY{o}{=} \PY{l+s+s2}{\PYZdq{}}\PY{l+s+s2}{NA}\PY{l+s+s2}{\PYZdq{}}
\end{Verbatim}
\end{tcolorbox}

    \begin{tcolorbox}[breakable, size=fbox, boxrule=1pt, pad at break*=1mm,colback=cellbackground, colframe=cellborder]
\prompt{In}{incolor}{20}{\boxspacing}
\begin{Verbatim}[commandchars=\\\{\}]
\PY{l+s+sd}{\PYZsq{}\PYZsq{}\PYZsq{}}
\PY{l+s+sd}{УПРАЖНЕНИЕ 14}
\PY{l+s+sd}{Объедините фреймы данных survey\PYZus{}data\PYZus{}decoupled и фрейм данных видов данных путем добавления соответствующей информации о видах (название, класс, царство и т. д.) к отдельным наблюдениям. }
\PY{l+s+sd}{Назначьте выходные данные новой переменной survey\PYZus{}data\PYZus{}species.}
\PY{l+s+sd}{\PYZsq{}\PYZsq{}\PYZsq{}}

\PY{n}{survey\PYZus{}data\PYZus{}species} \PY{o}{=} \PY{n}{pd}\PY{o}{.}\PY{n}{merge}\PY{p}{(}\PY{n}{survey\PYZus{}data\PYZus{}decoupled}\PY{p}{,} \PY{n}{species\PYZus{}data}\PY{p}{,} \PY{n}{how}\PY{o}{=}\PY{l+s+s2}{\PYZdq{}}\PY{l+s+s2}{left}\PY{l+s+s2}{\PYZdq{}}\PY{p}{,} \PY{n}{left\PYZus{}on}\PY{o}{=}\PY{l+s+s2}{\PYZdq{}}\PY{l+s+s2}{species}\PY{l+s+s2}{\PYZdq{}}\PY{p}{,} \PY{n}{right\PYZus{}on}\PY{o}{=}\PY{l+s+s2}{\PYZdq{}}\PY{l+s+s2}{species\PYZus{}id}\PY{l+s+s2}{\PYZdq{}}\PY{p}{)}
\PY{n}{survey\PYZus{}data\PYZus{}species} \PY{o}{=} \PY{n}{survey\PYZus{}data\PYZus{}species}\PY{o}{.}\PY{n}{drop}\PY{p}{(}\PY{p}{[}\PY{l+s+s2}{\PYZdq{}}\PY{l+s+s2}{species\PYZus{}x}\PY{l+s+s2}{\PYZdq{}} \PY{p}{,} \PY{l+s+s2}{\PYZdq{}}\PY{l+s+s2}{species\PYZus{}id}\PY{l+s+s2}{\PYZdq{}} \PY{p}{]}\PY{p}{,} \PY{n}{axis} \PY{o}{=} \PY{l+m+mi}{1}\PY{p}{)}
\PY{n}{survey\PYZus{}data\PYZus{}species} \PY{o}{=} \PY{n}{survey\PYZus{}data\PYZus{}species}\PY{o}{.}\PY{n}{rename} \PY{p}{(}\PY{n}{columns} \PY{o}{=} \PY{p}{\PYZob{}} \PY{l+s+s2}{\PYZdq{}}\PY{l+s+s2}{species\PYZus{}y}\PY{l+s+s2}{\PYZdq{}} \PY{p}{:} \PY{l+s+s2}{\PYZdq{}}\PY{l+s+s2}{species}\PY{l+s+s2}{\PYZdq{}} \PY{p}{\PYZcb{}}\PY{p}{)}    
\PY{n}{survey\PYZus{}data\PYZus{}species}\PY{o}{.}\PY{n}{head}\PY{p}{(}\PY{p}{)}
\end{Verbatim}
\end{tcolorbox}

            \begin{tcolorbox}[breakable, size=fbox, boxrule=.5pt, pad at break*=1mm, opacityfill=0]
\prompt{Out}{outcolor}{20}{\boxspacing}
\begin{Verbatim}[commandchars=\\\{\}]
   plot verbatimSex  wgt                       datasetname     sex  \textbackslash{}
0     2           M  NaN  Ecological Archives E090-118-D1.    male
1     3           M  NaN  Ecological Archives E090-118-D1.    male
2     2           F  NaN  Ecological Archives E090-118-D1.  female
3     7           M  NaN  Ecological Archives E090-118-D1.    male
4     3           M  NaN  Ecological Archives E090-118-D1.    male

   occurrence\_id  eventDate      genus   species    taxa
0              1 1977-07-16        NaN       NaN     NaN
1              2 1977-07-16        NaN       NaN     NaN
2              3 1977-07-16  Dipodomys  merriami  Rodent
3              4 1977-07-16  Dipodomys  merriami  Rodent
4              5 1977-07-16  Dipodomys  merriami  Rodent
\end{Verbatim}
\end{tcolorbox}
        
    \begin{tcolorbox}[breakable, size=fbox, boxrule=1pt, pad at break*=1mm,colback=cellbackground, colframe=cellborder]
\prompt{In}{incolor}{21}{\boxspacing}
\begin{Verbatim}[commandchars=\\\{\}]
\PY{l+s+sd}{\PYZsq{}\PYZsq{}\PYZsq{}}
\PY{l+s+sd}{УПРАЖНЕНИЕ 15}
\PY{l+s+sd}{Прочитайте файл excel \PYZsq{}plot\PYZus{}location.xlsx\PYZsq{} и сохраните данные как переменную plot\PYZus{}data с 3 столбцами: plot, xutm, yutm.}
\PY{l+s+sd}{\PYZsq{}\PYZsq{}\PYZsq{}}

\PY{n}{location\PYZus{}filename} \PY{o}{=} \PY{l+s+s2}{\PYZdq{}}\PY{l+s+s2}{..}\PY{l+s+se}{\PYZbs{}\PYZbs{}}\PY{l+s+s2}{..}\PY{l+s+se}{\PYZbs{}\PYZbs{}}\PY{l+s+s2}{data}\PY{l+s+se}{\PYZbs{}\PYZbs{}}\PY{l+s+s2}{003}\PY{l+s+se}{\PYZbs{}\PYZbs{}}\PY{l+s+s2}{plot\PYZus{}location.xlsx}\PY{l+s+s2}{\PYZdq{}}
\PY{n}{plot\PYZus{}data} \PY{o}{=} \PY{n}{pd}\PY{o}{.}\PY{n}{read\PYZus{}excel}\PY{p}{(}\PY{n}{location\PYZus{}filename}\PY{p}{,} \PY{n}{skiprows}\PY{o}{=}\PY{l+m+mi}{3}\PY{p}{,} \PY{n}{usecols}\PY{o}{=}\PY{p}{[}\PY{l+s+s2}{\PYZdq{}}\PY{l+s+s2}{plot}\PY{l+s+s2}{\PYZdq{}}\PY{p}{,} \PY{l+s+s2}{\PYZdq{}}\PY{l+s+s2}{xutm}\PY{l+s+s2}{\PYZdq{}}\PY{p}{,} \PY{l+s+s2}{\PYZdq{}}\PY{l+s+s2}{yutm}\PY{l+s+s2}{\PYZdq{}}\PY{p}{]}\PY{p}{)}
\PY{n}{plot\PYZus{}data}\PY{o}{.}\PY{n}{head}\PY{p}{(}\PY{p}{)}
\end{Verbatim}
\end{tcolorbox}

            \begin{tcolorbox}[breakable, size=fbox, boxrule=.5pt, pad at break*=1mm, opacityfill=0]
\prompt{Out}{outcolor}{21}{\boxspacing}
\begin{Verbatim}[commandchars=\\\{\}]
   plot           xutm          yutm
0     1  681222.131658  3.535262e+06
1     2  681302.799361  3.535268e+06
2     3  681375.294968  3.535270e+06
3     4  681450.837525  3.535271e+06
4     5  681526.983040  3.535281e+06
\end{Verbatim}
\end{tcolorbox}
        
    \begin{tcolorbox}[breakable, size=fbox, boxrule=1pt, pad at break*=1mm,colback=cellbackground, colframe=cellborder]
\prompt{In}{incolor}{22}{\boxspacing}
\begin{Verbatim}[commandchars=\\\{\}]
\PY{l+s+sd}{\PYZsq{}\PYZsq{}\PYZsq{}}
\PY{l+s+sd}{УПРАЖНЕНИЕ 16}

\PY{l+s+sd}{Примените преобразование функции pyproj к plot\PYZus{}data, используя столбцы xutm и yutm, }
\PY{l+s+sd}{и сохраните результат в двух новых столбцах, называемых decimalLongitude и decimalLatitude:}

\PY{l+s+sd}{Создайте функцию transform\PYZus{}utm\PYZus{}to\PYZus{}wgs, }
\PY{l+s+sd}{которая берет строку DataFrame и возвращает серию из двух элементов с долготой и широтой.}

\PY{l+s+sd}{Протестируйте эту функцию в первой строке plot\PYZus{}data.}
\PY{l+s+sd}{Теперь примените эту функцию ко всем строкам (используйте правильный параметр оси)}
\PY{l+s+sd}{Назначьте результат предыдущего шага столбцам decimalLongitude и decimalLatitude.}
\PY{l+s+sd}{\PYZsq{}\PYZsq{}\PYZsq{}}

\PY{n}{transformer} \PY{o}{=} \PY{n}{Transformer}\PY{o}{.}\PY{n}{from\PYZus{}crs}\PY{p}{(}\PY{l+s+s2}{\PYZdq{}}\PY{l+s+s2}{EPSG:32612}\PY{l+s+s2}{\PYZdq{}}\PY{p}{,} \PY{l+s+s2}{\PYZdq{}}\PY{l+s+s2}{epsg:4326}\PY{l+s+s2}{\PYZdq{}}\PY{p}{)}

\PY{k}{def} \PY{n+nf}{transform\PYZus{}utm\PYZus{}to\PYZus{}wgs}\PY{p}{(}\PY{n}{row}\PY{p}{)}\PY{p}{:}
    \PY{k}{return} \PY{n}{transformer}\PY{o}{.}\PY{n}{transform}\PY{p}{(}\PY{n}{row}\PY{p}{[}\PY{l+s+s2}{\PYZdq{}}\PY{l+s+s2}{xutm}\PY{l+s+s2}{\PYZdq{}}\PY{p}{]}\PY{p}{,} \PY{n}{row}\PY{p}{[}\PY{l+s+s2}{\PYZdq{}}\PY{l+s+s2}{yutm}\PY{l+s+s2}{\PYZdq{}}\PY{p}{]}\PY{p}{)}

\PY{n}{plot\PYZus{}data}\PY{o}{.}\PY{n}{iloc}\PY{p}{[}\PY{l+m+mi}{0}\PY{p}{:}\PY{l+m+mi}{1}\PY{p}{]}\PY{o}{.}\PY{n}{apply}\PY{p}{(}\PY{n}{transform\PYZus{}utm\PYZus{}to\PYZus{}wgs}\PY{p}{,} \PY{n}{axis}\PY{o}{=}\PY{l+m+mi}{1}\PY{p}{)}
\end{Verbatim}
\end{tcolorbox}

            \begin{tcolorbox}[breakable, size=fbox, boxrule=.5pt, pad at break*=1mm, opacityfill=0]
\prompt{Out}{outcolor}{22}{\boxspacing}
\begin{Verbatim}[commandchars=\\\{\}]
0    (31.938851000000383, -109.08282899999641)
dtype: object
\end{Verbatim}
\end{tcolorbox}
        
    \begin{tcolorbox}[breakable, size=fbox, boxrule=1pt, pad at break*=1mm,colback=cellbackground, colframe=cellborder]
\prompt{In}{incolor}{23}{\boxspacing}
\begin{Verbatim}[commandchars=\\\{\}]
\PY{c+c1}{\PYZsh{} plot\PYZus{}data[[\PYZdq{}decimalLongitude\PYZdq{},\PYZdq{}decimalLatitude\PYZdq{}]] = plot\PYZus{}data.apply(transform\PYZus{}utm\PYZus{}to\PYZus{}wgs, axis=1).tolist()}
\PY{n}{plot\PYZus{}data}\PY{p}{[}\PY{p}{[}\PY{l+s+s2}{\PYZdq{}}\PY{l+s+s2}{decimalLongitude}\PY{l+s+s2}{\PYZdq{}}\PY{p}{,}\PY{l+s+s2}{\PYZdq{}}\PY{l+s+s2}{decimalLatitude}\PY{l+s+s2}{\PYZdq{}}\PY{p}{]}\PY{p}{]} \PY{o}{=} \PY{n}{plot\PYZus{}data}\PY{o}{.}\PY{n}{apply}\PY{p}{(}\PY{k}{lambda} \PY{n}{row}\PY{p}{:} \PY{n}{transformer}\PY{o}{.}\PY{n}{transform}\PY{p}{(}\PY{n}{row}\PY{p}{[}\PY{l+s+s2}{\PYZdq{}}\PY{l+s+s2}{xutm}\PY{l+s+s2}{\PYZdq{}}\PY{p}{]}\PY{p}{,} \PY{n}{row}\PY{p}{[}\PY{l+s+s2}{\PYZdq{}}\PY{l+s+s2}{yutm}\PY{l+s+s2}{\PYZdq{}}\PY{p}{]}\PY{p}{)}\PY{p}{,} \PY{n}{axis}\PY{o}{=}\PY{l+m+mi}{1}\PY{p}{)}\PY{o}{.}\PY{n}{tolist}\PY{p}{(}\PY{p}{)}

\PY{n}{plot\PYZus{}data}
\end{Verbatim}
\end{tcolorbox}

            \begin{tcolorbox}[breakable, size=fbox, boxrule=.5pt, pad at break*=1mm, opacityfill=0]
\prompt{Out}{outcolor}{23}{\boxspacing}
\begin{Verbatim}[commandchars=\\\{\}]
    plot           xutm          yutm  decimalLongitude  decimalLatitude
0      1  681222.131658  3.535262e+06         31.938851      -109.082829
1      2  681302.799361  3.535268e+06         31.938887      -109.081975
2      3  681375.294968  3.535270e+06         31.938896      -109.081208
3      4  681450.837525  3.535271e+06         31.938894      -109.080409
4      5  681526.983040  3.535281e+06         31.938970      -109.079602
5      6  681599.189293  3.535294e+06         31.939078      -109.078836
6      7  681224.809581  3.535180e+06         31.938113      -109.082816
7      8  681332.659397  3.535157e+06         31.937884      -109.081680
8      9  681406.168037  3.535155e+06         31.937859      -109.080903
9     10  681482.625940  3.535174e+06         31.938017      -109.080091
10    11  681556.670465  3.535180e+06         31.938056      -109.079307
11    12  681630.880607  3.535198e+06         31.938203      -109.078519
12    13  681246.131957  3.535060e+06         31.937028      -109.082613
13    14  681320.391953  3.535065e+06         31.937054      -109.081827
14    15  681395.165904  3.535066e+06         31.937059      -109.081036
15    16  681469.975430  3.535072e+06         31.937094      -109.080244
16    17  681548.306618  3.535076e+06         31.937117      -109.079415
17    18  681622.221792  3.535078e+06         31.937126      -109.078633
18    19  681689.773453  3.535114e+06         31.937438      -109.077912
19    20  681476.480274  3.534987e+06         31.936334      -109.080191
20    21  681551.229556  3.535001e+06         31.936448      -109.079398
21    22  681626.500362  3.535002e+06         31.936441      -109.078602
22    23  681698.098229  3.535039e+06         31.936763      -109.077838
23    24  681704.204436  3.535238e+06         31.938560      -109.077736
\end{Verbatim}
\end{tcolorbox}
        
    \begin{tcolorbox}[breakable, size=fbox, boxrule=1pt, pad at break*=1mm,colback=cellbackground, colframe=cellborder]
\prompt{In}{incolor}{24}{\boxspacing}
\begin{Verbatim}[commandchars=\\\{\}]
\PY{l+s+sd}{\PYZsq{}\PYZsq{}\PYZsq{}}
\PY{l+s+sd}{УПРАЖНЕНИЕ 17}
\PY{l+s+sd}{Извлеките только столбцы для присоединения к нашему набору данных опроса: }
\PY{l+s+sd}{идентификаторы участков, decimalLatitude и decimalLongitude в новую переменную с именем plot\PYZus{}data\PYZus{}selection.}
\PY{l+s+sd}{\PYZsq{}\PYZsq{}\PYZsq{}}
\PY{n}{plot\PYZus{}data\PYZus{}selection} \PY{o}{=} \PY{n}{plot\PYZus{}data}\PY{o}{.}\PY{n}{loc}\PY{p}{[}\PY{p}{:}\PY{p}{,} \PY{p}{[}\PY{l+s+s2}{\PYZdq{}}\PY{l+s+s2}{plot}\PY{l+s+s2}{\PYZdq{}}\PY{p}{,} \PY{l+s+s2}{\PYZdq{}}\PY{l+s+s2}{decimalLongitude}\PY{l+s+s2}{\PYZdq{}}\PY{p}{,} \PY{l+s+s2}{\PYZdq{}}\PY{l+s+s2}{decimalLatitude}\PY{l+s+s2}{\PYZdq{}}\PY{p}{]}\PY{p}{]}
\PY{n}{plot\PYZus{}data\PYZus{}selection}
\end{Verbatim}
\end{tcolorbox}

            \begin{tcolorbox}[breakable, size=fbox, boxrule=.5pt, pad at break*=1mm, opacityfill=0]
\prompt{Out}{outcolor}{24}{\boxspacing}
\begin{Verbatim}[commandchars=\\\{\}]
    plot  decimalLongitude  decimalLatitude
0      1         31.938851      -109.082829
1      2         31.938887      -109.081975
2      3         31.938896      -109.081208
3      4         31.938894      -109.080409
4      5         31.938970      -109.079602
5      6         31.939078      -109.078836
6      7         31.938113      -109.082816
7      8         31.937884      -109.081680
8      9         31.937859      -109.080903
9     10         31.938017      -109.080091
10    11         31.938056      -109.079307
11    12         31.938203      -109.078519
12    13         31.937028      -109.082613
13    14         31.937054      -109.081827
14    15         31.937059      -109.081036
15    16         31.937094      -109.080244
16    17         31.937117      -109.079415
17    18         31.937126      -109.078633
18    19         31.937438      -109.077912
19    20         31.936334      -109.080191
20    21         31.936448      -109.079398
21    22         31.936441      -109.078602
22    23         31.936763      -109.077838
23    24         31.938560      -109.077736
\end{Verbatim}
\end{tcolorbox}
        
    \begin{tcolorbox}[breakable, size=fbox, boxrule=1pt, pad at break*=1mm,colback=cellbackground, colframe=cellborder]
\prompt{In}{incolor}{25}{\boxspacing}
\begin{Verbatim}[commandchars=\\\{\}]
\PY{l+s+sd}{\PYZsq{}\PYZsq{}\PYZsq{}}
\PY{l+s+sd}{УПРАЖНЕНИЕ 18}
\PY{l+s+sd}{Объедините DataFrame plot\PYZus{}data\PYZus{}selection и DataFrame survey\PYZus{}data\PYZus{}decoupled }
\PY{l+s+sd}{путем добавления соответствующей информации о координатах к отдельным наблюдениям с помощью функции pd.merge(). }
\PY{l+s+sd}{Назначьте выходные данные новой переменной survey\PYZus{}data\PYZus{}plots.}
\PY{l+s+sd}{\PYZsq{}\PYZsq{}\PYZsq{}}

\PY{n}{survey\PYZus{}data\PYZus{}plots} \PY{o}{=} \PY{n}{pd}\PY{o}{.}\PY{n}{merge}\PY{p}{(}\PY{n}{survey\PYZus{}data\PYZus{}decoupled}\PY{p}{,} \PY{n}{plot\PYZus{}data\PYZus{}selection}\PY{p}{,} \PY{n}{how}\PY{o}{=}\PY{l+s+s2}{\PYZdq{}}\PY{l+s+s2}{left}\PY{l+s+s2}{\PYZdq{}}\PY{p}{,} \PY{n}{left\PYZus{}on}\PY{o}{=}\PY{l+s+s2}{\PYZdq{}}\PY{l+s+s2}{plot}\PY{l+s+s2}{\PYZdq{}}\PY{p}{,} \PY{n}{right\PYZus{}on}\PY{o}{=}\PY{l+s+s2}{\PYZdq{}}\PY{l+s+s2}{plot}\PY{l+s+s2}{\PYZdq{}}\PY{p}{)}
\PY{n}{survey\PYZus{}data\PYZus{}plots} \PY{o}{=} \PY{n}{survey\PYZus{}data\PYZus{}plots}\PY{o}{.}\PY{n}{rename} \PY{p}{(}\PY{n}{columns}\PY{o}{=}\PY{p}{\PYZob{}} \PY{l+s+s1}{\PYZsq{}}\PY{l+s+s1}{plot}\PY{l+s+s1}{\PYZsq{}} \PY{p}{:} \PY{l+s+s1}{\PYZsq{}}\PY{l+s+s1}{verbatimLocality}\PY{l+s+s1}{\PYZsq{}} \PY{p}{\PYZcb{}}\PY{p}{)}

\PY{n}{survey\PYZus{}data\PYZus{}plots}\PY{o}{.}\PY{n}{head}\PY{p}{(}\PY{p}{)}
\end{Verbatim}
\end{tcolorbox}

            \begin{tcolorbox}[breakable, size=fbox, boxrule=.5pt, pad at break*=1mm, opacityfill=0]
\prompt{Out}{outcolor}{25}{\boxspacing}
\begin{Verbatim}[commandchars=\\\{\}]
   verbatimLocality species verbatimSex  wgt  \textbackslash{}
0                 2     NaN           M  NaN
1                 3     NaN           M  NaN
2                 2      DM           F  NaN
3                 7      DM           M  NaN
4                 3      DM           M  NaN

                        datasetname     sex  occurrence\_id  eventDate  \textbackslash{}
0  Ecological Archives E090-118-D1.    male              1 1977-07-16
1  Ecological Archives E090-118-D1.    male              2 1977-07-16
2  Ecological Archives E090-118-D1.  female              3 1977-07-16
3  Ecological Archives E090-118-D1.    male              4 1977-07-16
4  Ecological Archives E090-118-D1.    male              5 1977-07-16

   decimalLongitude  decimalLatitude
0         31.938887      -109.081975
1         31.938896      -109.081208
2         31.938887      -109.081975
3         31.938113      -109.082816
4         31.938896      -109.081208
\end{Verbatim}
\end{tcolorbox}
        
    \begin{tcolorbox}[breakable, size=fbox, boxrule=1pt, pad at break*=1mm,colback=cellbackground, colframe=cellborder]
\prompt{In}{incolor}{26}{\boxspacing}
\begin{Verbatim}[commandchars=\\\{\}]
\PY{n}{survey\PYZus{}data\PYZus{}plots}\PY{o}{.}\PY{n}{to\PYZus{}csv}\PY{p}{(}\PY{l+s+s2}{\PYZdq{}}\PY{l+s+s2}{interim\PYZus{}survey\PYZus{}data\PYZus{}species.csv}\PY{l+s+s2}{\PYZdq{}}\PY{p}{,} \PY{n}{index}\PY{o}{=}\PY{k+kc}{False} \PY{p}{)}
\end{Verbatim}
\end{tcolorbox}


    % Add a bibliography block to the postdoc
    
    
    
\end{document}
